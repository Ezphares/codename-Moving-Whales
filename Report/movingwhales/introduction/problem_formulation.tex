\section{Problem formulation}
%TODO: srsly? problem formulation? cracks me up
When developing a system at a large scale like a social music management system,
one opportunity is to do it one man. 
%TODO: NOPE! chuck testa.
But when you are in a group it is more efficient to split the problem into 
smaller sub problems and assign a sub problem to each group member. 
There may arise several different problems when developing such a system, 
like how to make a system that is a music management system but at the same 
time it has to be a social networking system.
%TODO: omformuleres
The problem may be divided into two large problems, 
a management part and a social part. And then those parts can then be further 
divided into smaller problems and at the end it all gets combined.
%TODO: sprog + rund cirkel
Also when making such a system you always have to keep in mind that you are 
developing for other people, also referred to as users.
%TODO: jeg troede vi blev enige om at det her skulle rettes?
So in that way you always have to think of making the system as user friendly 
as possible. To achieve that one has go through some tests with the users 
involved. 