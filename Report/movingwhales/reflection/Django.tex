\subsection{Django}
Our use of Django meant that we did not have to worry about creating and managing the database, which was a huge advantage
seeing as none of us had any previous experience with these tasks, as we have not yet had any courses in database design. The
rapid prototyping in Django meant that progress happened very fast, and it was very motivational having a prototype ready
early in the development process. The extensive, singular documentation for Django also meant that novice users had a place to look things up when they got stuck in the code. 
The debug server provided in Django was an extremely useful tool in the development process, seeing as code could be tested
while writing it. Compiling on the fly meant that non-working code could rapidly be fixed, and the result of any changes were instantly visible.

The Python programming language was fairly straightforward to learn which meant that development could start quickly, which in
turn meant that there was time for more SCRUM sprints before we had to stop to focus on finishing the report. 

A downside of Django, albeit one we did not experience is, that when you need something from the database that the ORM does not provide, the custom SQL needed to do whatever you want to do is hard to integrate with the SQL from the ORM. As mentioned however, this was not a problem for us so all in all using Django for web development was an extremely positive experience
and the group will definitely consider using it for future web development projects.