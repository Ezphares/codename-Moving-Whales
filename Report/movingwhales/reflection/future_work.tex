\section{Future work}
While the application has a lot of functionality as it is, it is not ``done'' and probably not ready
for release.

The issues that should most likely be worked on first, are the known bugs, and lack of visual feedback
in the interface, especially when getting friend requests, and more instructions to the user, like
where to find the session flyout.

The next thing to work on, would be creating playlists, since most of the code is already there, the
user interface for it is written, and it just needs to be glued together.

The last thing strictly necessary before a release is user interface elements to delete ``stuff''.
In its current form, there is no option for the user to delete songs from his library, delete playlists
or delete friends. And while the last may avoid a bit of drama, it should always be possible for
a user to delete things he added himself.

With the essential features done, it would be time to start on some of the features which enhances
the user experience. A good place to start would be messaging system for users, a song rating
system, and a Facebook-like event system, to help arrange music sessions.

At last the software could be expanded with some features which, while interesting, is not really
a part of what the system was made for. These includes OAuth and various music statistics.

During all of this, a general streamlining of the user interface, and optimization of the server and
database should be taking place, so that introducing new features does not mke the interface seem
cluttered, or slow the existing functionality down.