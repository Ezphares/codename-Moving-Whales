\subsection{Pair Programming}
The use of pair programming was introduced at the very beginning of the project. It was seen as a good way for the group
members, who all had different skill levels concerning programming, to learn from each other. However pair programming was not
employed correctly by all team memebers, and that meant that pair programming was, in fact, not experienced by all team
members. 

The more skilled programmers felt weighed down by the novices, and the novices felt trampled by the more skilled
programmers, which led to the group soon abandoning the technique in favour of working individually and asking for help from
each other when encountering problems. In hindsight this might have been a bad idea, although with the effort put into coding
by some people it may not have made a difference after all, seeing as pair programming requires people to commit to it in
order to work. Pair programming also takes a few weeks to get used to, according so some studies \cite{cockburn00}, which means that the group
might have discarded the technique prematurely. Pair programming may be employed in future projects, however with the strict
requirement that people commit to it like a motherfucker.