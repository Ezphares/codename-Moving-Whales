\subsection{Scrum}
Using an agile development method definitely had both upsides and downsides. It took a lot of self-discipline to keep using the practices in Scrum even though they sometimes felt like an utter waste of time and energy. We also spent some time at first getting to know the SCRUM method before starting, which cut down on time for development.
In the end we used the following practices:
\begin{itemize}
	\item \textbf{Sprints} The team worked with development in sprints of one week's length. Each week work tasks were divided
	among the group members and at the end of each sprint the finished code was demonstrated in front of the other group
	members. The success of the sprints is arguable. Some group members were very diligent and delivered quality code on time
	every time, while others either did not deliver any code at all, were late or commited untested buggy code to the project.
	
	All in all, however the principle of working in sprints was deemed quite effective, seeing as it, when it works, produces a
	product that can be compiled without errors at the end of each sprint, which is great for demonstrational purposes. This
	means that the schedule is flexible, i.e. a sprint can be added or detracted withouth it leaving the group with an
	unfinished product that cannot be demonstrated.
	
	\item \textbf{Product Backlog} The product backlog, i.e. the need-want-nice list, worked out quite well, helping the group
	keep the important goals in mind when deciding what to write in the sprint backlog for next sprint. Having the features
	suggested in a list kept them organized and meant that people with less skill could be assigned to non-crucial
	assignments from the want- and nice-lists.
	
	\item \textbf{Sprint Planning Meeting} The quality of these meetings varied greatly from sprint to sprint. During some
	sprints they were effective and constructive, providing the group members with a clear definition of each task and leaving
	them knowing exactly what to do. During other sprints the meetings were unstructured, unfocused and left people frustrated,
	demoralized and tired. From time to time it seemed like people did not take the meetings seriously, and just wanted to get
	their tasks and go home.
	
	\item \textbf{Sprint Review} The sprint review was not used to its full potential, as some group members rarely had
	functioning code to show, which cut them quite short. The meetings in themselves were however a motivational factor, keeping
	people working, knowing that they had to present the code to the other group members at  the end of the sprint.
	
	\item \textbf{Sprint Retrospective} These were definitely the most taxing meetings, often resulting in long, unconstructive
	discussions and petty arguments. At best they were short and uneventful, at worst long, frustrating and damaging to the
	group dynamic.
	
	\item \textbf{Roles} We did not make as much use of the roles of SCRUM as we could have, which was a shame, seeing as a
	product owner might have been good to have, to keep people working.
	
	\item \textbf{Daily Scrum} The daily scrum meetings were employed for one sprint, but they were quickly abandoned, seeing
	as people did not show up in the group room every day to work, and they were hard to schedule with the course calendar not
	being the same each week. The week when daily scrum was tested people were late for the meetings, and the objective of the
	meetings, i.e. asking for help and reporting progress was not put to full use. Instead people asked for help outside of the
	meetings. However, using the meetings as a motivator worked well for some group members.
	
	\item \textbf{Planning Poker} The use of planning poker was introduced late in the project, but it was definitely a success,
	giving people a feel of how hard it is to accurately estimate the size of a task. It also led to some very good discussions
	about the tasks at hand, clearing up misconceptions and misunderstandings along the way. 
	
	\item \textbf{Burndown Chart} The burndown chart was introduced at the beginning of the project, however, seeing as people
	preferred to work at home, the chart was not updated daily and soon it was deemed unfit for the project.
\end{itemize}

All in all using Scrum correctly proved quite hard, especially when people did not fulfill the tasks given and seeing as the courses of the semester were unevenly distributed, which made some meetings hard to schedule with a proper spacing. 