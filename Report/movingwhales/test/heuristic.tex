\subsection{Participatory heuristic evaluation}
This method of testing involves the participants, alongside usability experts, as ``work-domain experts''. The same procedure as for the expert version is applied, and the participants are briefed about the requirements. An expert heuristic evaluation relies on human computer interaction (HCI) experts to perform the evaluation, based on the following design principles:
\begin{enumerate}
	\item Visibility \textit{Making sure that people can see what functions are available, and what the system is doing.}
	\item Consistency \textit{Be consistent in design features and be consistent with similar systems and standard ways of working.}
	\item Familiarity \textit{Use language and symbols that the intended audience are familiar with or provide a suitable metaphor to help ease of learning.}
	\item Affordance \textit{Design things so it is clear what they are for.}
	\item Navigation \textit{Provide support for navigating the system, such as directional and information signs.}
	\item Control \textit{Make it clear who or what is in control and allow people to take control.}
	\item Feedback \textit{Give rapid feedback from the system so people know how their actions are effecting the system.}
	\item Recovery \textit{Enable quick and effective recovery from unintended actions like mistakes.}
	\item Constraints \textit{Make sure people are not able to do things they should not do.}
	\item Flexibility \textit{Allow things to be done in multiple ways, so that people with different levels of experience can use it.}
	\item Style \textit{Make it look good.}
	\item Conviviality \textit{Make the system polite and pleasant to use.}
\end{enumerate}
As this is not the method chosen, heuristic evaluation will not be further explained.