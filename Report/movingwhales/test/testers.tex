\subsection{Participant evaluation}
The 3 participants were between 19 and 28 of age, two had moderate computer skills and one had extensive computer skills and programming experience. They took between 11 and 16 minutes in completing the tasks given.
\begin{itemize}
	\item The first task was very clear to all participants, although one commented that the bio box was too small. Two of the participants got the date-format for birthday wrong, seeing as it was in the format YYYY-MM-DD instead of the format known to most Danes, YYYY-DD-MM, which produced a known bug preventing them from completing the other tasks before they had created a new profile with correct format. The last participant was informed about the date-format to save time. One participant provided a password that was too short, but when prompted with an error message quickly changed it.
	\item The second task was completed without any trouble for all participants.
	\item The third task was also quite easy, although two participants complained about lacking support for pressing the enter-button, and so they had to click on the button.
	\item Accepting a friend request proved harder than expected for all participants. One skipped the task at first but later returned to it and one started pressing Chrome-buttons instead of buttons on the webpage. Two participants completed the task, both stating that it would have been nice with some sort of notification showing that you had a friend request.
	\item The upload screen got mixed reviews. One participant said it was very nice, one complained that the drag'n drop uploading was hard to figure out and two out of three participants did not read the helptext on the homepage before starting upload. One participant tried uploading to the community screen, but quickly realized that it didn't work and two out of three dropped the music in the wrong part of the homepage. All participants got it to work though.
	\item All participants tried doubleclicking and selecting and pressing play before realising that you had to drag the song to the play for it to playback. All participants overlooked the helptext telling them to drag music to the player to play at first but in the end everyone got it to work.
	\item All participants completed this seventh and eigth tasks of skipping through the song and sorting their libraries without any problems.
	\item The community search did not prove problematic to any of the participants, they all got that they had to go to community to search for people, although two participants complained that they lacked enter-button support. The participant who skipped the third task found the friend request and accepted it, and then searched and added the Test-profile to their profile.
	\item By far the hardest task was the session invitation. Two out of three participants took a good while and some help to figure out how to invite people, and complained that the system was cryptic. One participant had seen the invitation flyout when she moused over the session sidebar and thus had no problems, but one complained that the sidebar lacked an invite button in addition to the drag'n drop option. Chat went without any trouble for all participants, although websocket crashed when one participant send the session invite, so at first chat was not available.
	\item The final task was completed without any trouble for all participants.
\end{itemize}

When interviewed about the prototype and the test afterwards all participants agreed that the tasks were overall quite easy and very realistic, except for the session invite which was unintuitive and hard to figure out, unless you happend upon the invitation flyout by chance. No participants felt distracted by giving a comment while doing the tasks. 

The best parts of the prototype were the drag'n drop upload, according to one participant, and the ease of use once you had gotten used to it according to the other two. 

The worst was the lack of notifications in regards to friend requests, the lack of enter-support and the unstable state of the application.