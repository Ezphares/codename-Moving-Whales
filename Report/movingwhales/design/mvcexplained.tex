\subsection{Model-View-Controller Explained}

\noindent
\textbf{Model}

\noindent
The model manages the behavior and data of the application domain. Furthermore, the model objects responds to requests for information about its state (from the view usually). The model also responds to instructions to change state (usually from the controller)\cite{modelviewcontroller}.

\vspace{5 mm}
\noindent
\textbf{View}

\noindent
Views are the aspect of the program that is visible to the user (the UI)\cite{mvcasp}. The user interface is created from the model data\cite{mvcasp}. An edit-view would be dependent on the model to supply it with database entries (or as the user would view it, white-textboxes), in which data could be added. 

\vspace{5 mm}
\noindent
\textbf{Controller}

\noindent
The controller interprets the user's keyboard and mouse input and informs the model or view to change accordingly\cite{mvcasp}.