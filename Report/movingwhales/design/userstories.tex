\section{User stories}

``User stories are the real-world experiences, ideas, anecdotes and knowledge of people''\cite{Benyon10}. The user stories in this section are fictional, seeing as the time for interviews was limited. The stories are, however, based on own experiences, and have been included to shed some light on the potential problems that the system would help solve.

\vspace{5 mm}
\textbf{Poul 23 years old}

\vspace{5 mm}
\noindent
Weekend was approaching and I wanted to invite over some friends to start a party. I was pleased that people seemed eager to attend. Come weekend most of my invitations were accepted by the people I invited. I did not really plan much for the party, I never do. Usually the custom is that my friends will bring with them what they wish to drink during the night which means that the only things I usually have to take care of, are my own needs and a single ice-breaker drink each guest will receive when they are all assembled.  

As far as music goes, I like to keep a single laptop in the corner of the space of my living room where we are all assembled so people can use youtube to find a specific song they want. This is typically an easy way of setting up some musical entertainment, but I do not like it very much. When people go to use the computer to find music they typically stop the song currently playing and skip right ahead to the next song. One of my friends took offense by this as the song that was playing prior to the switch was his favorite tune, and to him it felt like his taste in music was not accepted by my other friend. This caused some unnecessary friction during the party I would much rather have been without.

\vspace{5 mm}
\noindent
\textbf{Anna 27 years old}

\vspace{5 mm}
\noindent
I had just gotten home from work when I decided to turn on my computer and talk to my friends. I logged onto Facebook to see if anything new had happened, and I noticed that a friend had sent me a link to a song he really liked. I clicked on it and was taken to youtube where the song would play. I really liked it and I came to think that I actually knew a lot of songs that sounded sort of like it. So I started a chat with him and we began to exchange music links back and forth. It would definitely have been easier if there was a way for the two of us to be able to listen to the same song simultaneously. 

It would be nice if we did not have to look up youtube links that matched. It was especially annoying because the songs that interested my friend were in my music library, so when youtube was being too poor on quality I had to send him the files over MSN instead of facebook. So what could have been a music discussion drowned in finding the correct program to use, to send files through. I did not feel comfortable either because since I had actually purchased my CD's I felt like he kind of mooched on the money I had spent on my music.

\vspace{5 mm}
\noindent
\textbf{Erik 19 years old}

\vspace{5 mm}
\noindent
I was going to my friend's place after class to play some videogames. I had remembered to bring my external hard drive on which I keep my music library so it could easily be plugged in when we played. It felt annoying though to have to install iTunes on his computer and then find a way to certify all the songs I have in my library, before it would work out. It took a lot of work and patience but eventually we managed to listen to the music I keep on my hard drive. Sometimes I also worry that my hard drive will take damage - if I fall or I lose my bag, which makes me nervous and insecure. I try to look out for my things, but if I damage my hard drive I would lose all of my music which does make me feel uneasy. Unfortunately it is the only way though for me to transport my music in a way that is kind of mobile.