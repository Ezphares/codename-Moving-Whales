\section{The Heuristic method}
When we speak of the Heuristic method then we talk about using a number of different methods fore evaluation of a HCI(Human Computer Interaction) design. When one wants to make an evaluation of design, one can have a list of heuristics(also called design principles)to begin with. Here are some examples on some design principles:
\begin{enumerate}
\item Visibility
\item Consistency
\item Navigation
\item Control
\item Constraints
\item Style
\item Conviviality
\end{enumerate}
and a lot more.

\textbf{Visibility:} is about making things visible so that people can see what options they have and what the system is currently doing. An important part of the visibility is the psychological principle that states, it is easier to recognize things than to have to recall them. Visibility is not always about making things visible for the eye it is also about making things visible through the use of sound and touch.\\
\textbf{Consistency:} makes the usability more friendly and easier to learn a new system. It is important to stay consistent so that users do not get more confused then necessary. There are two types of consistency. There is conceptual consistency and than there is physical consistency. Conceptual consistency is about ensuring the mappings are consistent. Physical consistency is about ensuring consistent behaviours and consistent use of colours, names, layouts and so on.\\
\textbf{Navigation:} is about providing functions to the user such the he/she can move around in the system: maps, directional signs and information signs.\\ 
\textbf{Control:} it should be obvious who is in control, the user or the system. To enhance the controls there should be clear, logical mapping controls and the effect that they have.\\
\textbf{Constraints:} should always be provided in any system such that the user does not do any inappropriate operations. People should be prevented from making serious errors.\\ 
\textbf{Style:} Designs should be stylish and attractive. It should keep up with the time and be in vague.\\ 
\textbf{Conviviality:} polite, friendly and generally pleasant software is more catchy than unpleasant software. When making interactive systems it is mandatory to take this in mind fore else if not considered one may end up with some product that loses audience.\\

This is not just a method to evaluate the finished system, it is also a method that can be used under development of a system. To help one keep in mind how to make a system more usable.\\

When those principles and more have been evaluated, one can get a pretty good picture of the whole system before launching it or starting to test it on users.\\

To move the evaluation of the design of a system further we get to a point where people/future users have to be brought into the evaluation. They have to help one understand their needs, desires and how they are using the system such that developers can redesign some features if needed.\\

When all this is said it does not mean that this is the only approach for evaluating design. There are lots of other methods and principles for evaluation of system designs. This method has not been chosen to test users, we will hear more about that later in Chapter 5, but it has been kept in mind under the development of the system.\cite{Benyon10} 

\subsection{Heuristic analysis of Movingwhales}

When looking at the design principles listed in ``The Heuristic method´´ and using those principles fore analysing Movingwhales we get a good look at if the system lives up to some of the principles. We can see that Movingwhales is at least living up to some of the principles.

If we start at visibility, we see fore example that buttons are visible. The colours are not clashing, there has been focused on not to use too many colours such that it would not become too confusing. It is also possible to see which functions and operations are connected together because they are sort of merged together in tables.

You can also see that consistency has been considered when the system has been under development. Fore example you can see that when changed between the different windows there are only small changes and it is in the big table. Colour usage has also been consistent and the design of different features  is also consistent, fore example the design of Playlist, Session and Chat in the table at the right side.

Navigation should not be the biggest problem cause there is not so much to navigate to at the current state of the system.

Most of the control is left to the user. Users manage their own music, playing and sessions they choose self whom to befriend with.

Of course there are also constraints such as it is not possible to upload files other than mp3 files. It is neither possible to create a user with the same user name or e-mail address.

You can say that style has also been though about in a way, fore example the whole style seems round fancy. Nowadays styling features a lot of roundness. And when you look at the top of the page, you can see that the player and profile info fields have been moved a bit forward so it gives and impression of 3D effect.

Conviviality:
