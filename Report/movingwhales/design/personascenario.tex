\section{Personas and scenarios}

Personas help designers envision whom it is they are designing for. This helps designers recognize that they are not designing for themselves. Personas are fictional, but it is important for designers to bestow a lot of human nature onto these personas to reflect reality as much as possible. The advantage of personas is the fact that it makes designers have to focus on something concrete, as opposed to a wide array of different possibilities.

Scenarios are stories about people undertaking activities in contexts using technologies. They are used to make designers focus on events in which they use the technology in development. Scenarios can be rich in detail and share information about the user that might seem trivial. Ultimately a scenario revolves around the technology being designed.\cite{Benyon10}
Below is an example of a persona, and a scenario.

\vspace{5 mm}
\noindent

\textbf{John}
\begin{itemize}
	\item Age 24.
	\item University student.
	\item No children and girlfriend.
	\item Lives alone.
	\item Active social life.
	\item Experienced computer-user.
\end{itemize}

\vspace{5 mm}
\textbf{John's scenario}
\begin{enumerate}
	\item It is Friday night and John just finished writing a very difficult report segment for his university report.
	\item John decides to take a break from the university stress.
	\item He decides he wants to listen to music before he heads to bed.
	\item He opens his browser and logs into ``Moving Whales''
	\item John is notified that four of his friends are online, which makes him happy.
	\item John decides to invite them all into a chat-session where they start talking.
	\item One of his friends, Steve, takes control of the session's playlist and puts on a bit of music. The music was not something John wanted to listen to, so he initiates a vote to skip to the next song in the playlist.
	\item Unfortunately John's vote was not accepted so John turns down his speakers and goes to fetch a drink.
	\item When he comes back to his computer he is pleased to see that the song has ended, at which point he turns up the volume again and begins to search for other songs via the program's functionality to add to the playlist.
	\item The program, based on his and his friends' music library, suggests three songs which he adds to the playlist.
\end{enumerate}