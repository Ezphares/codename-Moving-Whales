\subsection{Waterfall Model or Scrum?}

Scrum offers a lot of different practices that are nice to 
follow during the project development to increase 
communication. The group decided that using Scrum's 
practices could only be beneficial to ensure that each 
member in the team has ample room to communicate with
the rest of the team. The many meetings during a sprint when 
using Scrum only helps with 
creating a strong basis for communication.


Planning poker is one such practice, and is a card game 
played to put estimates
on assignments chosen for the sprint. The game is played 
during the Sprint Planning Meeting.
The game helps group members give an estimate on when a 
given task can be completed. Another
purpose of planning poker is to start a discussion to ensure 
that all group members throw
their estimates on the same basis - basically, that everyone 
understood the task correctly.


These many meetings and practices of Scrum seems like a helpful tool to keep the project
progressing. The many ``excuses'' for communication seems helpful to the project 
group. Scrum offers a safety net under the group and a framework to fall back 
on. The group decided to use Scrum for this project because many of the group
members had previous experience with trying to develop in Scrum. The last attempt using Scrum turned out a failure so the group
wanted to see if they could make it work this time around.


The waterfall model is initially quite easy to understand. There is not much that would come
across as overly surprising to a software development group. The principal of always
progressing forward down the phase tree, and never upward, is very comprehensible. The
problem with the waterfall model is however that the requirements must be specified very
strongly, very early in the development phase - there is no room for adding further features
as your knowledge expands. Therefore the waterfall model is sequential and SCRUM is
not\cite{waterfallexplained}. 


The group chose Scrum for this project due to the fact they are under education and are working on
a project. Using Scrum would give an opportunity to put new knowledge, attained through
various courses, to use in the project as it progressed\cite{waterfallvsagile11}. 
Furthermore, with Scrum it would be possible to adapt and throw in last-minute features that may have been
forgotten early on, but prove very important in the later stage. If the waterfall model was used, in theory, a new project should be started to add in 
the new features and functionalities of the system\cite{waterfallexplained}. 


An agile development method seems like the ideal choice, for when different 
possibilities are explored and it is uncertain how much of the end-goal can be 
managed in terms of the system's functionalities\cite{waterfallvsagile11}.
The group is of the understanding that Scrum is used in the industry which is another motivational factor to give it a chance.