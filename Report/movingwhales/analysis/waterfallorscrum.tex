\subsection{Waterfall Model or Scrum?}

Scrum offers a lot of different practices that are nice to follow during the project
development to increase communication. The group decided that using Scrum's practices could
only be beneficial to ensure that each member in the team has ample room to communicate with
the rest of the team. The many meetings during a sprint when using Scrum only helps with 
creating a strong basis for communication.


Planning poker is one such practice, and is a card game played to put estimates
on assignments chosen for the sprint. The game is played during the Sprint Planning Meeting.
The game help group members give an estimate on when a given task can be completed. Another
purpose of planning poker is to start a discussion to ensure that all group members throw
their estimates on the same basis - basically, that everyone understood the task correctly.
It also gives an idea of who can solve it quickly, and who would need more time.


These many meetings and practices of Scrum seemed like a helpful tool to keep the project
progressing. The many 'excuses' for communication seemed helpful to the project group. Scrum
therefore offers a safety net under the group and a framework to fall back on. Using the
waterfall model the project group would need to set up its own practices and use those -
whereas Scrum was a final package that with some education could be used right out of the
box so to speak. 


The waterfall model is initially quite easy to understand. There is not much that would come
across as overly surprising to a software development group. The principal of always
progressing forward down the phase tree, and never upward, is very comprehensible. The
problem with the waterfall model is however that the requirements must be specified very
strongly, very early in the development phase - there is no room for adding further features
as your knowledge expands. Therefore the waterfall model is sequential and SCRUM is
not\cite{waterfallexplained}. 


We chose SCRUM for this project due to the fact we are under education as we are working on
our project. Using SCRUM would give us an opportunity to put new knowledge, given through
various courses, to use in the project as we worked\cite{waterfallvsagile11}. Furthermore,
with SCRUM we would be able to adapt and throw in last-minute features that may have been
forgotten early on, but prove very important in the later stage. If we used the waterfall
model, we should in theory start a new project to add in the new features and
functionalities of a program\cite{waterfallexplained}. 


An agile development form seems like the ideal choice, for when you are exploring different
possibilities and not really certain how much of the end-goal can be managed in terms of the
program's functionalities\cite{waterfallvsagile11}.
The group is of the understanding that Scrum is used in the industry which is another
motivational factor to give it a chance.