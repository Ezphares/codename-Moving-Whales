\subsection{Waterfall Model or Scrum?}


The waterfall model is initially quite easy to understand. There is not much that would come across as overly surprising to a software development group. The principal of always progressing forward down the phase tree, and never upward, is very comprehensible. The problem with the waterfall model is however that the requirements must be specified very strongly, very early in the development phase - there is no room for adding further features as your knowledge expands. Therefore the waterfall model is sequential and SCRUM is not\cite{waterfallexplained}. 


We chose SCRUM for this project due to the fact we are under education as we are working on our project. Using SCRUM would give us an opportunity to put new knowledge, given through various courses, to use in the project as we worked\cite{waterfallvsagile11}. Furthermore, with SCRUM we would be able to adapt and throw in last-minute features that may have been forgotten early on, but prove very important in the later stage. If we used the waterfall model, we should in theory start a new project to add in the new features and functionalities of a program\cite{waterfallexplained}. 


An agile development form seems like the ideal choice, for when you are exploring different possibilities and not really certain how much of the end-goal can be managed in terms of the program's functionalities\cite{waterfallvsagile11}.
