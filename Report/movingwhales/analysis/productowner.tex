\subsubsection{The Product Owner}


The Product Owner is considered to be the 'customer' for whom the product is being developed.
The Product Owner is a well defined role in the Scrum framework. The product owner is in charge of the product backlog\cite{scrumguide11}. The product Backlog is an ordered list of all requirements needed for the product in development. The product Backlog is typically never finished and can be updated and adjusted at all times but only by the Product Owner\cite{scrumguide11}. Backlog management includes the following tasks as defined in\cite{scrumguide11}:

\begin{itemize}
	\item Clearly expressing backlog items.
	\item	Ordering the items in the product backlog to best achieve goals and missions.
	\item	Ensuring the value of the work the Development Team performs.
	\item	Ensuring the product backlog is visible and transparent (clear to all).
	\item	Make sure the product backlog shows what the Scrum Team will work on next.
	\item	Ensuring the development team understands items in the Product Backlog.
\end{itemize}


The Product Owner may give their task to the Development Team however, if that makes more sense in a given scenario\cite{scrumguide11}. The Product Owner will still be held accountable for the Product Backlog however - it is their responsibility regardless of who performs the task\cite{scrumguide11}.


The main thing to keep in mind is that the Product Owner is there to maximize the value of the product and the work of the Development Team\cite{scrumguide11}.  The Product Owner is a single person. A Product Owner is never multiple persons or an entire committee for that matter. It does not mean that the Product Owner cannot represent a committee or other organization\cite{scrumguide11}. The role cannot be maintained by multiple people\cite{scrumguide11}.


The Product Backlog determines how the Development Team works. Therefore it is important that only one person maintains the Product Backlog to ensure that the Development Team will not suddenly have to work under totally new requirements\cite{scrumguide11}.
