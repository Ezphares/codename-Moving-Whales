\subsubsection{The Product Owner}

The Product Owner is the customer for whom the product is being developed.
The Product Owner is in charge of the Product Backlog\cite{scrumguide11}. The
Product Backlog is an ordered list of all requirements needed for the product in
development. The Product Backlog is typically never finished and can be updated 
and adjusted at all times but only by the Product
Owner\cite{scrumguide11}. Backlog management includes the following tasks:

\begin{itemize}
	\item Clearly expressing backlog items.
	\item	Ordering the items in the Product Backlog to best achieve goals and missions.
	\item	Ensuring the value of the work the Development Team performs.
	\item	Ensuring the Product Backlog is visible and transparent.
	\item	Make sure the Product Backlog shows what the Scrum Team will work on next.
	\item	Ensuring the Development Team understands the Product
	Backlog.\cite{scrumguide11}
\end{itemize}

The Product Owner may give the task of managing the Product Backlog to the
Development Team, if that makes more sense in the given
scenario\cite{scrumguide11}. However The Product Owner will still be held
 accountable for the Product Backlog - it is their responsibility regardless of who
performs the task\cite{scrumguide11}.


The main thing to keep in mind is that the Product Owner is there to maximize the value of
the product and the work of the Development Team\cite{scrumguide11}.  The Product Owner is a
single person. A Product Owner is never multiple persons or an entire committee for that
matter. It does not mean that the Product Owner can not represent a committee or
other organization\cite{scrumguide11}.


The Product Backlog determines how the Development Team works. Therefore it is important
that only one person maintains the Product Backlog to ensure that the Development Team will
not suddenly have to work under totally new requirements\cite{scrumguide11}.