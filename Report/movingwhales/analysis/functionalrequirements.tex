\subsection{Functional and non-functional requirements}

Functional requirements are requirements of the system's functionality and behaviour. For instance, if it is a necessity for the system to function properly, that a high level of security is added and all communication between the client and server makes use of an encryption protocol, then that would be a functional requirement\cite{SEF01}.
The functional requirements of the program are all geared toward a finished first-release of the program. The group has prioritized getting the system up and running very high which can be shown in the need-want-nice list later in this section. 
The non-functional requirements are requirements that represent how the operation of the system has to be. 
As a general rule of thumb to describe the difference between the two: functional requirements are usually in the form of: "The system shall do <requirement>" and non-functional requirements are of the form: "system shall be <requirement>". So, a non functional requirement would be of the form: "how well should the program be encrypted".

With this in mind, the group created a need-want-nice list to specify the absolute necessary requirements, and the requirements to the program that were less necessary. A need-want-nice list helps illustrate and set into boxes what the actual requirements to the program are.


\subsubsection{Need}


\begin{itemize}
	\item Play music
	\item Upload music to library
	\item Sort the library
	\item Join a music session
	\item Synchronize music playing
	\item Simple search
\end{itemize}


The requirements in the need section, are the most important program requirements. These requirements are an absolute necessity to have completed before the program can be released.
\subsubsection{Want}


\begin{itemize}
	\item Easy to use 
	\item Joining libraries (with other people in a session)
	\item Manage playlists
	\item Weighted search
	\item Drag and drop functionality for the GUI
\end{itemize}


These requirements are less important than the ``need'' list in terms of
priority, but more important than the ``nice'' list.
\subsubsection{Nice}


\begin{itemize}
	\item Playing rights 
	\item Session chat
	\item DJ rating (up/downvote current DJ)
	\item Track rating
	\item Dynamic playlists
	\item Auto updating track metadata
	\item Avoiding duplicates
	\item Track comments
	\item Music discovery (find similar tracks based on other users preferences)
	\item Music statistics (e.g. number of times a song has been played) 
	\item Social events
	\item OAuth (login with facebook, twitter, google)
	\item Loading screen
\end{itemize}


The nice list is made up of features that are not necessary to have implemented in the program. This makes the nice-list the least prioritized list of requirements.
\subsubsection{Need-want-nice summarized}

\begin{center}
	\begin{tabular}{ | l | l | l | p{5cm} |}
	\hline
	\textbf{Need} & \textbf{Want} & \textbf{Nice} \\ \hline
	Play music & Easy to use & Playing rights \\ \hline
	Upload music & Unite libraries & Session chat \\ \hline
	Sort library & Manage playlists & DJ rating \\ \hline
	Join music session & Weighted search & Track rating \\ \hline
	Synchronize playing & Drag and drop& Dynamic playlists \\ \hline
	Simple search & & Auto update track meta\-data \\ \hline
	 & & Avoid duplicates\\ \hline
	 & & Track comments\\ \hline
	 & & Music discovery\\ \hline
	 & & Music statistics\\ \hline
	 & & Social events\\ \hline
	 & & OAuth \\ \hline
	 & & Loading screen\\ \hline
	\end{tabular}
\end{center}