\subsection{The Sprint}

The sprint is the key event of Scrum. 
A sprint is a container for all other events.
A sprint can be time boxed from anywhere between a few weeks, to as far as a month. 
There is no lower restriction as to how 
short a sprint can be, but a sprint's upper limit is a single calendar 
month\cite{scrumguide11}. The most important thing to note however is that 
when a sprint time box has been arranged, it will be kept throughout the 
entire product development period\cite{scrumguide11}. This means that a 
first sprint cannot last two weeks, and the second six weeks. There has to be 
consistency in the time box used. When a sprint has 
finished, a new one starts immediately afterwards\cite{scrumguide11}.


A sprint consists of five aspects, the Sprint Planning Meeting, Daily Scrums, 
development work, Sprint Review and the Sprint Retrospective\cite{scrumguide11}. 


When a sprint has begun it is important that no changes are made that would 
affect the Sprint Goal\cite{scrumguide11}. The Development Team composition 
remains fixed. Quality goals (non-functional requirements) do not 
decrease. The scope may be clarified however during a sprint and 
re-negotiated between the Product Owner and Development Team as more is 
learned\cite{scrumguide11}. 


A sprint may be canceled at any time by the Product Owner, if the current 
Sprint Goals are no longer relevant for the entirety of the product - for 
example, if the buyer simply wishes to go totally different ways with their 
product by excluding a feature currently being worked on\cite{scrumguide11}. 
This means that though during a sprint goals and aspirations cannot change - 
the entire sprint CAN be canceled at request of the Product 
Owner. When a sprint is canceled, a new sprint has to be 
set up\cite{scrumguide11}.

\subsubsection{Sprint Planning Meeting}

The work of a sprint is determined at the Sprint Planning Meeting. The entire Scrum Team is 
gathered at this meeting and the meeting is time boxed to eight hours for a one
month sprint. If a sprint is shorter than one month, the meeting does not have
to take eight hours\cite{scrumguide11}. 


The actual time box of this meeting can then decrease proportionally with the amount of time 
scheduled for a sprint. For example, if a sprint takes two weeks instead of four
weeks, the meeting should take four, instead of eight
hours\cite{scrumguide11}.  The Sprint Planning Meeting consists of two parts each time boxed to last half of the total time boxed duration 
for the Sprint Planning Meeting. In the event of an eight hour meeting, each
part would take four hours\cite{scrumguide11}.


The Sprint Planning Meeting should answer the following two questions:

\begin{itemize}
	\item What will be delivered in the increment at the end of this sprint?
	\item	How will the work needed to deliver the increment be
	achieved?\cite{scrumguide11}
\end{itemize}


\subsubsection{Part one: What will be done?}


In the first part of the meeting the Development Team tries to shed some light on what functionalities of the program they think reachable during the sprint\cite{scrumguide11}. In collaboration with the Product Owner's Product Backlog and past increments, the Development Team tries to construct a number of features and functionalities they think can be accomplished during the sprint. Only the Development Team may decide what features can be implemented. The goal of the first part of the meeting is a Sprint Goal\cite{scrumguide11}.
\subsubsection{Part two: How will it be done?}

The second part of the meeting is typically more technical and concrete. It is here the 
Development Team tries to figure out how to convert the requirements in the Product Backlog 
and the agreed functionalities they need to try and implement can actually be 
done\cite{scrumguide11}. 


The goal is to create a 'done' increment and have enough work during this sprint, that it 
will not be accomplished before the sprint's end\cite{scrumguide11}. On the other hand, if 
the Development Team feels that there is too much work to handle during the sprint, it can 
renegotiate the Sprint Goal to be more fitting\cite{scrumguide11}.


At the end of the Sprint Planning Meeting, the Development Team should be able to tell the 
Scrum Master and Product Owner how they intend to reach their Sprint Goal.
\subsubsection{The Sprint Review}

The Sprint Review is a meeting held at the end of a sprint. The meeting is time
boxed for four hours (less if the sprint is smaller), and is being held to
inspire feedback by the entire Scrum Team\cite{scrumguide11}. 
The meeting is expected to shed light on what is considered finished product,
and what is not\cite{scrumguide11}. 
The entire team then collaborates on figuring out what could be worked on for the next sprint\cite{scrumguide11}. 


Aside from discussing what was finished and what was not, the Development Team should also 
shed light on what went well during the sprint, and what did not go well\cite{scrumguide11}. 
Problems the Development Team ran into should be addressed as well - in
particular how these problems were solved\cite{scrumguide11}. 
The Development Team is expected to answer questions to the rest of the Scrum
Team regarding the product that was worked on during the sprint\cite{scrumguide11}. 
The result of the Sprint Review meeting should be a revised Product Backlog 
to form a foundation for the next Sprint Planning 
Meeting\cite{scrumguide11}.
\subsubsection{The Sprint Retrospective}

The Sprint Retrospective Meeting is a meeting that is time boxed for a three
hour duration on a one month sprint. The Sprint Retrospective starts after the
Sprint Review Meeting.
The purpose of the Sprint Retrospective Meeting is to allow the Scrum Team to look inward 
and inspect itself to find potential changes that would be beneficial to enact during the 
upcoming sprint\cite{scrumguide11}. The meeting seeks to inspect the 
following as defined by Ken Schwaber and Jeff Sutherland:

\begin{itemize}
	\item	How the last sprint went in regards to people, tools and processes. 
	\item Identity and order items that went well during the sprint.
	\item	Create a plan for implementing improvements to the way the Scrum Team
	works.\cite{scrumguide11}
\end{itemize}


The result of the Sprint Retrospective Meeting is a formal identification of improvements 
for the next sprint\cite{scrumguide11}. The Scrum Team does not have to use the Sprint 
Retrospective Meeting to find and adapt to new improvements - the meeting simply gives the 
Team a formal chance to sit down and inspect themselves for the
better\cite{scrumguide11}.
