\subsection{Pair Programming}
Pair programming is a coding technique used in agile development, where two programmers work together on one computer. 
One is the ``driver'' and writes the code, the other is the ``navigator'', and reviews the code as it is being written. 
The two switch every hour, or whenever they are comfortable. 
This might seem like a terrible idea at first, seeing as you have two people doing one man's work, 
but upon further inspection it turns out, that at a cost of only 15\% longer development-time 
you get less errors, more efficient code and happier staff \cite{cockburn00}. 

Programming in pairs forces the driver to explain his thoughts on the problem at hand, 
and not coding actively allows the navigator to see flaws in the code 
and focus on the solution of the next problem ahead of time. Having two people who know the code also
means, that if someone leaves the group, it will not take long to understand the code left behind.
The two programmers also learn from each other while programming, seeing as they can ask questions and show tips and tricks. Interviews state that even a pair of a novice programmer and an expert produces better code.\cite{cockburn00} However it is important to note, that both programmers have to care about the code being written, seeing as they have to keep their focus on the task at hand while coding.

Seeing as the group has a very different level of experience with programming, with some being complete beginners and some being seasoned programmers, pair programming could prove extremely useful in increasing the skill level of new programmers while producing better code. This is why the group has chosen pair programming as the programming technique.