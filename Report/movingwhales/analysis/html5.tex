\section{HTML5}
HTML5 is the newest revision of the HTML standard by W3C (World Wide Web Consortium). It is still a draft, and has been a work in progress since 2004, when research of contemporary implementations of HTML was started. 34\% of the 100 most popular websites already implement HTML5\cite{Maine11}, even though W3C states, that HTML5 will not become the standard until 2014.\cite{W3C11} This might, in part, be due to HTML5's ability to run on devices like tablets and smartphones, but also the fact that HTML5 offers APIs - something that Facebook has offered since the 20th of October 2011 \cite{Kelly11} - which allows people to rapidly create web applications. It has also made the DOM, an HTML4 afterthought, a fundamental part of the specification, 

HTML5 differs from its predecessors(like HTML4, the W3C Recommendation from 1997 \cite{Kesteren11}) in various areas. The HTML5 standard tries to implement many of the features that people used to have to resort to other languages to obtain. An example is the native drag and drop, a feature that, as the name implies, allows the user to drag an element and drop it elsewhere and thus trigger events on the webpage. This feature is native in HTML5, and is available on all objects simply by setting the ``draggable'' attribute to true. It has also implemented a new, native, custom attribute called ``data'', consisting of an attribute name prefixed with ``data-'', and an attribute value, which is given as a string. This allows embedding of custom data-attributes inside a given HTML element, which makes JavaScript simpler.\cite{Bewick10}. 

HTML5 seeks to support all the modern multimedia options, and has the new <video> and <audio> tags to help make multimedia easily implemented on the webpage. Another noteworthy new element is the <canvas> element, which allows the developer to script dynamic 2D graphics, useful for e.g. web games. 
It also supports scalable vector graphics (SVG), a specification used to preserve the quality of graphics when scaling them up and down.

The native drag and drop is very useful for uploads of files, seeing as the dragging and dropping of files is intuitive, even to new users. It allows for an interface without an upload button and thus frees up space for other uses. Drag and drop in a music player is also convenient when creating a playlist, or playing a song. Dragging the song to the list or onto the music player makes sense, seeing as the interaction is easily understood. 

\subsection{JavaScript}
%TODO lawl at the bellow line - H
%JavaScript is an object-oriented scripting language
JavaScript is a prototype-oriented scripting language usually used for
web-development. All major webbrowsers support JavaScript, and as it is the only
scripting language all major browsers support it is considered \textbf{the}
scripting language of the internet. 
JavasScript allows scripting of webpage events like keypresses and mouseclicks.
Since JavaScript has access to the DOM, it allows document manipulation,
providing users with an almost seamless interaction with the webpage.

%TODO extensive handing? non-user activated events? really? source?
%JavaScript also offers extensive handling of non-user activated events.
The most popular JavaScript library is jQuery, 
a library that, in compliance with their slogan, ``write less, do more.'', 
allows for faster scripting. It is currently used by 41.4\% of all
websites monitored by W3Techs \cite{W3Tech11} and is very useful for projects
needing rapid prototyping.

SoundManager 2 is a library which, as the name implies, lets the developer
manage and play sounds on webpages via a hidden Flash object. It consists of two
files: a JavaScript API and a Flash movie file. By providing a
JavaScript API to the functionality of the Flash file, developers don't need to
maintain any ActionScript (The scripting language Flash uses).
\cite{Schiller07}
