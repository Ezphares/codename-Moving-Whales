\section{HTML5}
HTML5 is the newest version of the HTML standard by W3C (World Wide Web
Consortium). It is still a draft, and has been a work in progress since 2004, 
when research of contemporary implementations of HTML was started. 
34\% of the 100 most popular websites already implement HTML5\cite{Maine11}, 
even though W3C states, that HTML5 will not become the standard until 2014.\cite{W3C11} This might, in part, 
be due to HTML5's ability to run on devices like tablets and smartphones, 
but also the fact that HTML5 offers some new APIs with features that ease the
development and allows people to rapidly create web applications.
HTML5 makes use of the latest DOM version, originally an HTML4
afterthought, that allowed the browser and scripts to dynamically access and
update the content and structure of documents\cite{H�garet05}.


HTML5 differs from its predecessors(like HTML4, the W3C Recommendation from 1997 \cite{Kesteren11}) in various areas. 
The HTML5 standard attempts to implement many of the features that people used
to have to resort to other languages to obtain.
An example is the native support for drag and drop, a feature that - as the name
implies - allows the user to drag an element and drop it elsewhere, thus
triggering events on the webpage.
This feature is native in HTML5, and is available on all objects simply by setting the ``draggable'' attribute to true. 
It also implements a new custom attribute called ``data'' 
consisting of an attribute name prefixed with ``data-''. This allows embedding
of custom data-attributes inside an HTML element, which simplifies many
scripted tasks.\cite{Bewick10}.


HTML5 seeks to support all the modern multimedia options, 
and has the new <video> and <audio> tags to help make multimedia easily implemented on the webpage. 
Another noteworthy new element is the <canvas> element, 
which allows the developer to script dynamic graphics, useful for e.g.
webbasedgames. 
It also supports scalable vector graphics (SVG), an image type where the image
consists of vector data instead of the usual bitmap data. Vector graphics allow
scaling of the image without loosing quality.


The native drag and drop is very useful for uploading files, 
seeing as the dragging and dropping of files is intuitive, even to new users.
%TODO source!? - H
It allows for an interface without an upload button and thus frees up space for other uses. 
Drag and drop in a music player is also convenient when creating playlists, or
playing songs.
Dragging the song to the list or onto the music player makes sense, seeing as the interaction is easily understood. 
%TODO again, source. maybe cite our test section if this were our findings
\subsection{JavaScript}
%TODO lawl at the bellow line - H
%TODO lawl at the above line, bellow, really? - B
%JavaScript is an object-oriented scripting language
JavaScript is a prototype-oriented scripting language usually used for
web-development. All major webbrowsers support JavaScript, and as it is the only
scripting language all major browsers support it is considered \textbf{the}
scripting language of the internet. 
JavasScript allows scripting of webpage events like keypresses and mouseclicks.
Since JavaScript has access to the DOM, it allows document manipulation,
providing users with an almost seamless interaction with the webpage.

%TODO extensive handing? non-user activated events? really? source?
%JavaScript also offers extensive handling of non-user activated events.
