\subsection{Non-functional Requirements Prioritized}

The non-functional requirements of the system will be rated on a four-step scale
from very important, important, less important to not important.
\\
Following every non-functional requirement and its rating, will be a short description of
what the non-functional requirement covers, as well as a justification behind the rating
given.
\newpage

\noindent
Usability - \textbf{Important}

\vspace{5 mm}
\noindent
\textit{Ease of use and learnability.}

\vspace{5 mm}

\noindent
It is in the nature of the program to be easy to use. A social, online music management
program should be easy for its users to understand and make use of, to ensure they see the
idea behind it, and keep using it frequently. Usability does not receive top rating
however, given that the project's main focus was music management.

\vspace{5 mm}
\noindent
Availability - \textbf{Important}

\vspace{5 mm}
\noindent
\textit{The proportion of time a system is in a functioning condition.}

\vspace{5 mm}
\noindent
Though it is important that a user has access to their music library, it should not be
rated \textit{very important}. It is not very important, because the program
should be used for recreational purposes, it would not be destructive for the program to be down during
scheduled maintenances. It should not be tagged \textit{less important} however,
given that too much disregard of uptime would make users look elsewhere for a different service.

\vspace{5 mm}
\noindent
Effectiveness - \textbf{Less important}

\vspace{5 mm}
\noindent
\textit{The amount of computing resources and code required by a program to perform a
function.}

\vspace{5 mm}
\noindent
Effectiveness is not trivially solved, but it is not a priority that should take precedence
over availability or usability for that matter. It is not important for this project to
have fast code-execution to make it perform as intended.

\vspace{5 mm}
\noindent
Correctness - \textbf{Less important}

\vspace{5 mm}
\noindent
\textit{The degree of which a program satisfies its requirement specification.}

\vspace{5 mm}
\noindent
Correctness is tagged less important because in a university project, there is quite
restricted development time. It is better to have a functioning program with some of the
functionalities decided upon, than a finished program in terms of functionalities - that is
too bugged to pass a presentation.

\vspace{5 mm}
\noindent
Security - \textbf{Not important}

\vspace{5 mm}
\noindent
\textit{The extent of which the program is secure from misuse by unauthorized users.}

\vspace{5 mm}
\noindent
This is a prototype meant for a presentation, therefore security is less
relevant for this program.
\newpage
Scalability - \textbf{Very important}

\vspace{5 mm}
\noindent
\textit{The ability with which a system may gracefully adapt to a growing
amount of work.}

\vspace{5 mm}
\noindent
The program handles large quantities of data and therefore hardware should not be
considered a carte blanche to avoid scalability. Instead, it is better to make
the system able to gracefully adapt to an explosive rise in users if it wound up
popular.

\vspace{5 mm}
\noindent
Reusability - \textbf{Less important}

\vspace{5 mm}
\noindent
\textit{The degree to which the code can be used in other applications.}

\vspace{5 mm}
\noindent
Developers should not go out of their way to fulfill the reusability aspect of
the code. It is considered good conduct if reusability is kept in mind when
more ``general'' aspects of the system are coded and it can be implemented
without much extra work but it is far from an integral part of the system design.

\vspace{5 mm}
\noindent
Interoperability - \textbf{Less important}

\vspace{5 mm}
\noindent
\textit{Effort required to couple one system with another.}

\vspace{5 mm}
\noindent
It should be possible to add some functionalities of other systems (e.g. Facebook or
Twitter) into this application. Since we are offering a service, it would be
nice to release an API that could be used elsewhere. It is a prototype however
and should be considered future work.

\vspace{5 mm}
\noindent
Testability - \textbf{Important}

\vspace{5 mm}
\noindent
\textit{Effort required to test a system to ensure it performs as intended.}

\vspace{5 mm}
\noindent
The system is developed using the agile development method Scrum. For this
reason testability has to be rated ``important'', to ensure that developers keep in mind to write the
code in a way that makes it easy to test the correctness of the system as more
features are added.
