\subsection{WebSocket efficiency}


In a standard HTTP connection, there is a lot of header overhead\cite{lubbersgreco}. What the web socket does very well is stripping redundant data from the data that is actually relevant. Cutting off the excess fat, and there is a lot of it when one uses a polling solution to keeping a web application updated in real-time, can dramatically decrease unnecessary network traffic\cite{lubbersgreco}. Looking at the latency, if a server takes 50ms to send a message to the browser, the polling application would introduce extra latency because the connection would terminate, only to restart (taking another 50ms) - which during this time of re-establishing a connection means that the server cannot contact the browser with new data\cite{lubbersgreco}. The web socket never breaks communication with the server once opened - unless specified otherwise, and allows the server to send new data immediately as it receives it\cite{lubbersgreco}. Obviously it would still take 50ms for the server to send data to the browser via the web socket - but the real gain here is that the web socket will not have to send a request to the server to re-establish connection after it receives a message, like when polling is used\cite{lubbersgreco}.