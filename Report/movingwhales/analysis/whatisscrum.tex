\subsection{What is Scrum?}


Scrum is an iterative development method\cite{scrumguide11}.
Ken Schwaber and Jeff Sutherland developed scrum as a framework for developing 
and sustaining complex products\cite{scrumguide11}. Scrum has been defined by 
Ken Schwaber and Jeff Sutherland, as follows:


\textbf{Scrum (n):} \begin{quotation}
A framework within people can address complex adaptive 
problems, while productively and creatively delivering products of the highest 
possible value
\end{quotation}\cite{scrumguide11}.

With this in mind, there are several key features that defines 
scrum as defined by Ken Schwaber and Jeff Sutherland\cite{scrumguide11}.

\begin{itemize}
	\item Lightweight
	\item Simple to understand
	\item Extremely difficult to master
\end{itemize}


The Scrum framework consists of 4 key ingredients. First and foremost - the Scrum Team: the 
Scrum Team is basically anyone who works on the project. Each member in the Team has a 
role - they will be discussed later in this section\cite{scrumguide11}.
The second ingredient is the events. The events in Scrum are typically meetings. The 
details of the events will be discussed later. The third ingredient is artifacts. Artifacts 
are things that add transparency to the project. For example, the Product Backlog, the 
Sprint Backlog or the burndown chart\cite{scrumguide11}. The fourth ingredient is the rules 
of Scrum. These rules are imperative for the succes of Scrum as they are the glue that binds 
the framework together. Rules are generally the codex that must be adhered to. For example, 
how long a sprint is time boxed, how long the daily Scrum is time boxed and so on. The rules 
govern the practices of Scrum\cite{scrumguide11}.



\subsubsection{Scrum theory}

Scrum was founded on the basis of empirical process control theory\cite{scrumguide11}.

Empiricism asserts that knowledge comes primairily via experience and basing 
decisions on evidence or what is known\cite{scrumguide11}. Scrum uses the three pillars of 
empiricism, transparency, inspection and adaption\cite{scrumguide11}.


\subsubsection{Transparency}

Transparency is a generally defined term. Transparency implies openness, communication and
accountability. In Scrum it is to ensure that the very significant aspects of the process 
are visible to those in control of the outcome. Transparency is used 
to ensure that there is a shared common standard of communication so observers understand 
what they see\cite{scrumguide11}.

To give an example of transparency in Scrum:

\begin{itemize}
	\item A common understanding of \textit{done} must be shared by those doing the work and 
those receiving the end product\cite{scrumguide11}.
\end{itemize}
\subsubsection{Inspection}


Inspectors inspect artifacts to ensure that there are no undesired variances toward reaching the product goal\cite{scrumguide11}. It is important that inspections are never so frequent that they get in the way of work\cite{scrumguide11}. To ensure a good balance between frequent inspections and not getting in the way of work, inspections are the most beneficial when they are performed by skilled inspectors\cite{scrumguide11}.
\subsubsection{Adaption}


If an inspector determines that the process deviates outside acceptable limits, it is important to adapt and adjust the material being processed to avoid further deviation\cite{scrumguide11}. Failing to do so can in worst case make the resulting product totally unacceptable\cite{scrumguide11}.
There are several ways that an inspection and an adjustment can be made, during any of these activities will ensure a smooth transition:

\begin{itemize}
	\item Sprint planning meeting
	\item	Daily scrum
	\item	Sprint review
	\item	Sprint retrospective
\end{itemize}