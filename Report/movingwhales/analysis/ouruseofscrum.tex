\subsection{Our usage of Scrum}

Though the Scrum framework states that a version of Scrum that does not employ every process
is not in fact Scrum, we have for this project used Scrum as much as it made sense for us -
even if the end result cannot be labeled as achieved through Scrum\cite{scrumguide11}. 


The group set out to use all of the processes of Scrum (e.g. Sprint Meetings). 
We skipped the burndown chart as we would not always be at the university and it would hence
be difficult to update. We used planning poker during the Sprint Planning Meeting to get an
overview of thoughts and ideas for solving a project problem. Planning Poker helps estimate
when a given task could be solved. 


Planning Poker is a Scrum event based on a card game wherein each player has a set of cards
- each card representing a number or coffee break. These cards would then be used to put a
number on a task to measure their relative size. 


The group did not make use of a Product Owner. It was considered letting the project's
supervisor become the Product Owner. We decided in the group that it was better to merge the
roles a little bit, and give the Scrum Master the Product Backlog management job.