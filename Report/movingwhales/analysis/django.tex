\section{Choice of framework}
Choice of framework for developing is largely a matter of preference. There are multiple frameworks available in various languages, such as Django\cite{djangoproject11}  for developing in Python, CakePHP\cite{cake11} for developing in PHP and Ruby on Rails\cite{ruby11} for development in Ruby. These are all MVC-based frameworks, and the choice of framework thus comes down to what the group decided was the most efficient.

CakePHP was one option, but there is no singular documentation for the framework which would slow the work down significantly, as no group member had previous experience with the framework. Seeing as a few of the group members had programming experience with Python, and the Django framework was very well documented,the group chose Django as the framework for developing this project.

Django is made in the Python programming language, an object-oriented language, which runs on on either major platform (Windows, Linux / Unix and Mac OS)\cite{psf11}. While django is not exactly an MVC\cite{dalling09} it still incorporates Models, Views (which fill the role of the controller and half the role of the view in a standard MVC setup) and Templates (which fills the remaining part of the view-role), split in groups called ''applications��, so, depending on opinion, it offers the same benefits as an MVC.

Furthermore, Django uses an ORM (Object Relational Mapping)\cite{techtarget08} which automatically creates a database with the correct tables and colums from the model files, and allows all operations on the data in the database to be handled through Python class representations of the data. In a project like this, where almost everything in the database will be linked together, Django's ORM makes it very easy to traverse the links, as queries for related objects are made simply by referencing the correct attribute on the Python object.

It is crucial to the current project, that navigation on the site happens fluidly, without multiple page loads. This means there is going to be a lot of AJAX requests to the server, which each need to be handled in different ways. Django uses a very simple URL routing system, that makes it easy to pass these requests along to the correct views, and respond to them using a correct protocol.

Django also makes starting development quite easy by shipping with a simple webserver for debugging, with which it is possible to output anything to a console running on your machine, without also sending this data along with the response to the client. This helps debugging both the Django code, and the HTML/JavaScript, since you can get debug info without contaminating data.