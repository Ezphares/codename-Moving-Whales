\section{WebSocket}

WebSockets are a new technology meant to make it easier for developers who 
create web applications to keep their applications updated in real-time by the 
server\cite{andersen10}. A WebSocket is a full-duplex communication channel 
that operates via a socket\cite{lubbersgreco}. A WebSocket combines the 
standard HTTP-connection with more efficient means of transferring data and 
text\cite{lubbersgreco}.


WebSockets are very efficient as they decrease the latency during a message 
transfer significantly\cite{lubbersgreco}. When a browser visits a page it 
sends a HTTP request to the web server currently hosting the 
webpage\cite{lubbersgreco}. For data that is very sensitive to quick updates 
(a stock exchange for example) the data which the server sends could wind up 
being old and no longer relevant. Either because the user simply walks away 
from the computer for a set duration or the website's data is being constantly 
updated\cite{lubbersgreco}. The user would have to keep hitting the browser's 
refresh button to make sure that the data that is received is the newest data 
- obviously this is not a very clever solution\cite{lubbersgreco}. WebSockets 
help alleviate this problem.


There are a number of other solutions that try to solve the same issue the 
WebSocket solves - but bear in mind that these solutions are much older and 
among the first solutions to solve the problem of updating real-time 
data\cite{lubbersgreco}. These solutions can also be a study in their own and 
will therefore not be described extensively in this report-segment. Examples 
of those solutions are polling, long-polling and streaming.

%gawd sakes, Det hedder en WebSocket, ikke en Websocket, 
%og SLET ikke Web socket.


\subsection{Polling}

When polling is being used, it ``bombards'' the server with a lot of 
HTTP request packages, each checking if the server has new data available, if 
the server has no new data ready for the browser, then the connection 
terminates - but restarts again after a set interval. 
Polling is good when you know at which interval the data is typically 
being updated - basically, when you know when new data should be ready. If on the other hand you are not certain 
when new data is ready polling will inevitably make a lot of unnecessary 
requests to the server. Every time new data is available 
during a poll, the data is transferred from the server to the client after 
which the connection terminates and restarts.\cite{lubbersgreco}
\subsection{Long-polling}

Long-polling functions much like the way polling does. The intention is 
that, via long-polling, the browser creates a connection with the server 
and the server then keeps this connection running with the client for a 
longer duration. Unlike polling, requests are not sent 
repeatedly. The connection 
terminates if the server does not respond within the time-interval. 
After that, it resumes with a new long-lasting poll. 
If the server has a high message volume, long-polling does not differ much 
from traditional polling since each notification of new data returned to the 
client would terminate the connection - only to restart the 
procedure.\cite{lubbersgreco}
\subsection{Streaming}


The browser sends a complete request, but the server maintains an open response that is kept continually updated\cite{lubbersgreco}. The connection can either be kept open permanently or temporarily - there are no requirements one way or the other\cite{lubbersgreco}. The server updates the response every time there is data ready to be sent to the client\cite{lubbersgreco}. The server however, does not close off the connection after data has been sent\cite{lubbersgreco}. Other parameters have to be met here before that happens\cite{lubbersgreco}.
\subsection{WebSocket efficiency}


In a standard HTTP connection, there is a lot of overhead\cite{lubbersgreco}. 
What the web socket does very well is stripping redundant data from the data 
that is actually relevant, which there is a lot of, when a polling solution is 
used to keep a website updated in real-time\cite{lubbersgreco}. 

Looking at the latency, if a server takes 50ms to send a message to the 
browser, the polling application would introduce extra latency because the 
connection would terminate, only to restart (taking another 50ms) - which 
during this time of re-establishing a connection means that the server cannot 
contact the browser with new data\cite{lubbersgreco}. The web socket never 
breaks communication with the server once opened - unless specified otherwise, 
and allows the server to send new data immediately as it receives 
it\cite{lubbersgreco}. Obviously it would still take 50ms for the server to 
send data to the browser via the web socket - but the real gain here is that 
the web socket will not have to send a request to the server to re-establish 
connection after it receives a message, like when polling is 
used\cite{lubbersgreco}.
\subsection{WebSocket Support}

The WebSocket standard consists of two parts.
The first part is a protocol that is currently under a standardization 
procedure by the organization IETF (Internet Engineering Task Force) who 
cooperate closely with W3C. The second part is a 
JavaScript-API that is used on the browser side\cite{andersen10}. 
Because WebSockets are new, not all browsers support them, and not all 
browsers support the same WebSocket draft\cite{websocketprotocol11}.  For Opera WebSocket compatibility is disabled by 
default\cite{websocketprotocol11}, IE10 will supports WebSocket when released (Developer Preview supports hybi-10) and rfc-6455 is only supported by 
Google Chrome 16\cite{websocketecho}.