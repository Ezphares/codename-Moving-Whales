\subsubsection{The Development Team}


The Development Team are those who work on releasing a "Done" increment on the product at
the end of each sprint\cite{scrumguide11}. The Development Team is characterized by the
following:


\begin{itemize}
	\item Self-organizing. No one may tell the Development Team how to turn Product Backlog
into increments of 'done' product.
	\item	Cross-functional. They possess the necessary skills between them to create a product
increment.
	\item	Flat-hierarchy. The individual titles of the team are restricted to developer. No
one may be higher or lower in rank.
	\item	The entire team is held accountable. The Team will have members specialized in
specific, single areas of focus but accountability belongs to the Development Team as a
whole.
	\item	Development Teams do not contain sub-teams dedicated to particular domains (e.g.
testing, business analysis).
\end{itemize}


It is necessary for the Development Team to maintain their own work to ensure increased
efficiency and effectiveness\cite{scrumguide11}.
The size of the Development Team may vary. As a rule of thumb a Development Team should
remain small enough to stay nimble, but large enough to complete a significant portion of
work\cite{scrumguide11}. A Development Team should not be a large organ wherein coordination
would be near impossible\cite{scrumguide11}. Determining the size of the development team is
a balance act between making sure no skill constraints occur in the team, and that a
significant portion of work can be made\cite{scrumguide11}.