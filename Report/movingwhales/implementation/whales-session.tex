\section{application sessions on client}

One of the core features of our application is the session feature. A user can
invite any other user on his/her friendlist to join the users session. When the
user logs on, a new session is created and the user is placed in this session.
If the user invites another user to join the session, the invited user will
leave his/her own session and join the inviting user's session. This means that
the maximum number of session is equivalent to the number of users online.

\begin{snippet}[language=JavaScript,label=lst:whales.session structure,caption=The whales.session structure]
whales.session: {
    pending_accepts: []
    pending_invites: []
    errorHandlers: {} //a map of functions to handle errors
    successHandlers: {} // a map of functions to handle success
    init: function() //bind event and set up handler calls
    invite: function(username) //invite <username>
    accept: function(username) //accept invite from <username>
    reject: function(username) //reject invite from <username>
    leave: function() //leave current session
    friendsOnline: function() //sends a "who is online" request to the server
}
\end{snippet}

The client side session handling logic is placed in the
\lstinline$whales.session$ object. The structure is shown in listing \ref{lst:whales.session structure}.
The first two \verb$pending_*$ arrays in the structure holds incomming and
outgoing session invites. the \verb$*Handlers$ maps holds the handlers that is
run when a response is recived from the server concerning the session system.


All this is controlled in the \verb$init$ function seen in listing \ref{lst:whales.session.init}

\begin{snippet}[language=JavaScript,label=lst:whales.session.init,caption=The whales.session.init function] 
init: function() {
    $(window).bind("sync_session",function(e){
        var p = e.syncObject;
        if (p.payload.type === "success") {
            if(p.payload.handler in whales.session.successHandlers) {
                whales.session.successHandlers[p.payload.handler](p);
            } else {
                console.log("Could not find success handler");
            }
        } else if(p.payload.type === "error") {
            if(p.payload.handler in whales.session.errorHandlers) {
                whales.session.errorHandlers[p.payload.handler](p);
            } else {
                console.group("Could not find error handler");
                console.log(p);
                console.groupEnd();
            }
        } 
    });
}
\end{snippet}

Here we bind an event handler to the ``sync\_session'' event. A custom event
created in the ``whales.sync.js'' script. This event is triggered when a
package of the type ``session'' is recived from the server.


The event handler then checks the payload type for either ``success'' or
``error'' and fire the corresponding handler with the incomming data package as
the first argument.

\begin{snippet}[language=JavaScript,label=lst:incomming_event,caption=Handler for incomming events] 
incomming_invite: function(syncObject) {
    whales.session.pending_accepts.push(syncObject.payload);
    whales.loading(true,"Invite from "+syncObject.payload.data,1000);
    if($("#btn_session").hasClass("selected")) {
        nav.session(function(){
            whales.loading(false);
        });
    }
},
\end{snippet}

An example of an event handler would be the ``incomming\_invite'' handler listed
in \ref{lst:incomming_event}. The handler takes a syncObject as first argument
(the syncObject is just a standardised structure we use as communication
between the client and server).
The handler then pushes the payload of the syncObject on to the
``pending\_accepts'' array and calls 

\begin{snippet}[firstnumber=3]
whales.loading(true,"Invite from "+syncObject.payload.data,1000);
\end{snippet}

This function controls the loading indicator on the site. The first argument is
a boolean that controls the visibility of the loading indicator. The second
argument is a string containing a message to the user. The third argument is a
duration that controls for how long the message will be shown. If the third
argument is undefined, the loading indicator won't disappear until the next call
to ``whales.loading'' overrides the state.


Finally the handler checks whether or not the user is currently looking at the
``sessions'' page. If the ``sessions'' page is the currently shown page, the
handler reloads the page (with the \lstinline$nav.session$ function). This will
refresh the current session view to make sure that new invites are also shown on
the page.
