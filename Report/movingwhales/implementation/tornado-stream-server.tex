\section{Streaming server}

The streaming server is, like the WebSocket server, written on top of Tornado
\cite{tornadoweb11}, allowing for quick and easy development.
We needed a file server to stream music to connected clients. As this is just
for development we wrote up a small and simple file server to do the
job\ref{lst:streamhandler}.

\begin{snippet}[language=Python,label=lst:streamhandler,caption=The streaming server] 
class StreamHandler(tornado.web.RequestHandler):
    def get(self,path):
        fullpath = os.getcwd()+"/tracks/"+path
        try:
            with open(fullpath,"rb") as f:
                bytes = f.read(1024)
                self.set_header("Content-Type",'audio/mpeg3');
                while bytes != "":
                    self.write(bytes)
                    self.flush()
                    bytes = f.read(1024)
            self.finish()
        except:
            self.set_status(404)
            self.set_header("Content-Type",'text/plain');
            self.write("could not find track")
            self.finish()
\end{snippet}

The class is mapped to a URL on the same port as the WebSocket server. This
avoids cluttering up ports for no good reason. Our Tornado Application is
defined as follows:

\begin{snippet}[language=Python,caption=The streaming server]
class Application(tornado.web.Application):
    def __init__(self):
        handlers = [
            (r"/sync/", ChatSocketHandler),
            (r"/file/([^/]+)", StreamHandler),
        ]
        settings = dict()
        tornado.web.Application.__init__(self, handlers, ** settings)
        self.lock = thread.allocate_lock()
        thread.start_new_thread(WhalesSessionManager.syncronize, (self.lock,))
\end{snippet}

Here we see the tornado Application definition. The ``handlers'' variable
defines two turples that serves as URL mappings. The first mapping - the
``/sync/'' url - is mapped to the misleadingly named class
``ChatSocketHandler'' (The original purpose of this class was merely chat
relay). The second mapping maps all URLs matching the regular expression
\verb$/file/([^/]+)$ to the StreamHandler defined in listing 
\ref{lst:streamhandler}. The regular expression translate to: ``any string 
with the character sequence `/file/' followed by 1 or more characters that is 
not `/'''.


The code shown in listing \ref{lst:streamhandler} provides an easy way of 
running tests on the system without having to deploy it with a full-scale file
server like Apache. For production purposes however, this simple file server
should be replaced with a better one like the Apache or Lighttpd server.

