\section{WebSocket Data Pipeline}

When a WebSocket is created, functions to handle different events may also be
defined. The possible events are:

\begin{snippet}[label=lst:websocket events]
onopen
onclose
onerror
onmessage
\end{snippet}

These are defined by settings the corresponding attribute on the created 
WebSocket object. This means that one function defines one action for one
Websocket object. This is fairly simple, but unpractical when building an
application where serveral parts needs to be able to communicate through the 
same WebSocket object.


To overcome this problem, we introduce an abstaction layer between the 
WebSocket and the rest of the application. For each of the events in listing
\ref{lst:websocket events} we define a function that gathers incomming data from
the event, and packs it into an event that is then triggered on the global
window object. An example of this can be found in listing
\ref{lst:websocket.onerror}.

\begin{snippet}[caption=The onerror websocket event,label=lst:websocket.onerror] 
this.socket.onerror = function(event){
    var e = $.Event('sync_error',{
        originalEvent:event
    });
    $(window).trigger(e);
};
\end{snippet}

By doing this we allow any other part of our application to bind to a specific
event on the window object and handle the WebSocket data. The naming convention
for the events is an event name prefixed with ``sync\_''.


The ``onmessage'' WebSocket event is triggered every time new data is avaiable
from the server. We handle this by - as with the other WebSocket events -
triggering an event on the window object, should any other part of the
application need direct access to incomming data. The ``onmessage'' also calls a
decoding function that decodes the raw string from the server into a javascript
object that is then parsed on to another function for further handling.
This object is our data package containing all the information from the server.

\begin{snippet}[caption=The onmessage WebSocket event,label=lst:websocket.onmessage]
this.socket.onmessage = function(event){
    var e = $.Event('sync_raw',{
        originalEvent:event
    });
    e.data = event.data;
    $(window).trigger(e);
    var syncObj = whales.sync.decode(event.data);
    whales.sync.handle(syncObj);
};
\end{snippet}

The definition of the ``onmessage'' event is seen in listing \ref{lst:websocket.onmessage}.
Line 2-6 forms an event called ``sync\_raw'' and triggers it on the window
object.
Line 7-8 calls the decode function on the event data, and then passes it along
to be handled by our sync handler as seen in listing \ref{lst:sync.handle}.

\begin{snippet}[caption=Handling of incomming data,label=lst:sync.handle]
    handle: function(obj) {
        //console.log("recived: "+obj.type);
        obj.time.recived = whales.sync.time.now(); // important!

        var deltaTransferTime = obj.time.recived - obj.time.sent;
        if(obj.from !== "server") { // we do not want to calc. avg. from server
            whales.sync.time.pushLocalOffset(deltaTransferTime);
        }
        
        var handledCorrect = null;
        if(obj.type in this.handlers) {
            handledCorrect = this.handlers[obj.type](obj);
        }
        
        if (handledCorrect === false || handledCorrect === null){
            var e = $.Event('sync_'+obj.type,{
                syncObject:obj
            });
            $(window).trigger(e);
        }
    }
\end{snippet}

The handler provides a much simpler way to work with incomming data. We allready
decoded the message in the WebSocket ``onmessage'' function. This handler then
adds a timestamp to the package, calculate the time between the package was sent
and recived (the client to client latency) and pushes it into an array that
automatically calculates the average for later use.


Then a check is made on an internal map of functions
(\lstinline$this.handlers$).
All packages contains a ``type'' property. If a handler with the
corresponding name is defined in the internal function map, this function is
called with the package object as it's argument. An example of one of these
internal handlers is shown in listing \ref{lst:sync.ping}


Finally a check is made to the return value of the internal function. If the
internal function returns true, it means that this package was handled
successfully and should be considered done.
Otherwise a new event is created with the ``type'' as name, prefixed with
``sync\_'' and triggered on the window object like other websocket events.

\begin{snippet}[caption=An internal package handler,label=lst:sync.ping]
ping: function (obj) {
    obj = whales.sync.validatedSyncObject(obj);
    //pong back
    obj.type = "pong"; // change object type to pong
    if(obj.from === "server") {
        obj.payload.client = whales.common.readcsrf();
    }
    obj.time.parsed = whales.sync.time.now();
    whales.sync.sendObject(obj);
    return true;
}
\end{snippet}

The above listing \ref{lst:sync.ping} shows the internal handler for ``ping''.
The ``ping'' is one of the simplest package types in our system. It checks that
the recived package is valid, changes the ``ping'' package to a ``pong''
package, makes sure that it is the server asking for a pong, and then
adds a payload to the ``pong'' package containing the clients secret
identification token to be identified on the server. ``pong'' packages will
never reach other clients, so transmitting the secret client token in a ``pong''
package does not pose a security risk. All ``pong'' packages are cough by the
server, and is used in latency calculations to clients.


In effect, this system provides a simple interface for receiving data from the
websocket. This speeds up development and reduces the amount of bugs that would
otherwise have arrised when introducing new code directly in to the
``onmessage'' websocket function.


