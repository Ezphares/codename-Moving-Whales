\section{WhalesWebsocket}

Since several browsers have implemented the websocket technology, making our
application as cross-browser as possible makes sense for any future work.
Unfortunatly, different browsers use different words for the same thing.


This is why we have choosen to wrap the WebSocket object in a wrapper object.
By doing this, we can ensure that current browser implementations with different
names, still functions as intended. This allso allows easy adding of new
implementations in other browsers. If Microsoft Internet Explorer were to
implement a version called IEWebSocket, that was still in compliance with the current
standards, we could very easily add this to our wrapper, and withing seconds
enable IE clients in our application. 


Further more it would be fairly easy to
implement a fallback if the browser did not allow for websockets. This would be
done by returning an object that would mimic the behavior of a websocket, but
using XHR requests and polling to achieve the same thing.


This is our WebSocket wrapper. The fallback object would replace the 
\lstinline$return null;$ on line 6.

 \begin{lstlisting}[language=JavaScript,caption=Our WebSocket abstraction object] 
 var WhalesWebSocket = function(adress) { if ("WebSocket" in window) { return new WebSocket(adress);
    } else if("MozWebSocket" in window) {
        return new MozWebSocket(adress);
    } else {
        return null;
    }
};
\end{lstlisting}

Currently this only supports the mozilla implementation and the standards
implementation of WebSocket. In time of writing, this works in Chrome, Safari
and Firefox.
%TODO: ^skal vi have kilde p� det, og evt. versionsnumre p�? - H