\section{WhalesWebsocket}

Since several browsers have implemented the WebSocket technology, working towards making the
application as cross-browser compatible as possible would be logical for future work.

Unfortunatly, different browsers use different keyword implementations for the same thing.

This is why the WebSocket object is wrapped in a wrapper object.
By doing this, It is ensured that current browser implementations with different
keywords, still function as intended. This also allows for easy adding of new
implementations in other browsers. If Microsoft Internet Explorer was to
implement a version called IEWebSocket, that would still be in compliance with the current
standards, and the group could very easily add this to the wrapper.

Further more it would be fairly easy to
implement a fallback if the browser did not allow for WebSockets. This would be
done by returning an object that would mimic the behavior of a WebSocket, but
using XHR (XML HTTP Request) and polling to achieve the same thing.


This is the WebSocket wrapper. 
The fallback object would replace the 
\lstinline$return null;$ on line 6.

 \begin{snippet}[language=JavaScript,caption=Our WebSocket abstraction object] 
 var WhalesWebSocket = function(adress) { if ("WebSocket" in window) { return new WebSocket(adress);
    } else if("MozWebSocket" in window) {
        return new MozWebSocket(adress);
    } else {
        return null;
    }
};
\end{snippet}

Currently this only supports the Mozilla implementation and the standard
implementation of WebSocket. At time of writing, this works in Chrome, Safari
and Firefox.