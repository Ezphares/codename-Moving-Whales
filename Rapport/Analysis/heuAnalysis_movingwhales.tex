\section{Heuristic analysis of Movingwhales}

When looking at the design principles listed in ``The Heuristic method´´ and using those principles fore analysing Movingwhales we get a good look at if the system lives up to some of the principles. We can see that Movingwhales is at least living up to some of the principles.\\
	If we start at visibility, we see fore example that buttons are visible. The colours are not clashing, there has been focused on not to use too many colours such that it would not become too confusing. It is also possible to see which functions and operations are connected together because they are sort of merged together in tables.\\
	You can also see that consistency has been considered when the system has been under development. Fore example you can see that when changed between the different windows there are only small changes and it is in the big table. Colour usage has also been consistent and the design of different features  is also consistent, fore example the design of Playlist, Session and Chat in the table at the right side.\\
	Navigation should not be the biggest problem cause there is not so much to navigate to.\\
	Most of the control is left to the user. Users manage their own music, playing and sessions they choose self whom to befriend with.\\ 
	Of course there are also constraints such as it is not possible to upload files other than mp3 files. It is neither possible to create a user with the same user name or e-mail address.\\
	You can say that style has also been though about in a way, fore example the whole style seems round. When styling today a lot of things get some roundness. When you look at the top of the page you can see that the player and profile info field have in a way been moved a bit forward so it gives and impression of 3D effect.
	Conviviality: %hmm??
%evt. indsæt billeder  